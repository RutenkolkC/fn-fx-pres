\documentclass{beamer}

\mode<presentation>
{
\usetheme{CambridgeUS}
\usecolortheme{dolphin}


%\setbeamercovered{transparent}
%\useinnertheme{circles}

\usepackage{pgfpages}
%\setbeameroption{show notes on second screen}

\setbeamertemplate{navigation symbols}{}
\setbeamertemplate{itemize subitem}[triangle]
}
\mode<handout>{\setbeamercolor{background canvas}{bg=black!5}}

%\usepackage{ngerman}
\usepackage[ngerman]{babel}
\usepackage[latin1]{inputenc}
\usepackage[T1]{fontenc}
\usepackage{lmodern}
\usepackage{amssymb}
\usepackage{listings}

\usepackage{color} 
\usepackage{microtype} 
\usepackage{graphicx} 
\usepackage{multimedia}
\usepackage[normalem]{ulem}


%better tabulars
\usepackage{array} 
\usepackage{booktabs}
\newcommand{\PreserveBackslash}[1]{\let\temp=\\#1\let\\=\temp}
\usepackage[NewCommands]{ragged2e} % Better raggedleft formating
\newcolumntype{v}[1]{>{\PreserveBackslash{\raggedright}\hspace{0pt}}p{#1}}
\renewcommand{\arraystretch}{1.20}
\usepackage{multirow}

%\usepackage{minted}

\definecolor{bg}{rgb}{0.95,0.95,0.95}

\title[fn-fx]{GUI-Programmierung mit fn-fx}
\author{Chris Rutenkolk} 
\institute[HHU]{Institut f�r Informatik\\Lehrstuhl f�r Softwaretechnik und Programmiersprachen}
\date[06.06.2019]{6. Juni 2019}
\logo{\pgfimage[width=2cm]{../bilder/hhulogo}}
