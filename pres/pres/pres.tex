\documentclass{beamer}

\mode<presentation>
{
\usetheme{CambridgeUS}
\usecolortheme{dolphin}


%\setbeamercovered{transparent}
%\useinnertheme{circles}

\usepackage{pgfpages}
%\setbeameroption{show notes on second screen}

\setbeamertemplate{navigation symbols}{}
\setbeamertemplate{itemize subitem}[triangle]
}
\mode<handout>{\setbeamercolor{background canvas}{bg=black!5}}

%\usepackage{ngerman}
\usepackage[ngerman]{babel}
\usepackage[latin1]{inputenc}
\usepackage[T1]{fontenc}
\usepackage{lmodern}
\usepackage{amssymb}
\usepackage{listings}

\usepackage{color} 
\usepackage{microtype} 
\usepackage{graphicx} 
\usepackage{multimedia}
\usepackage[normalem]{ulem}


%better tabulars
\usepackage{array} 
\usepackage{booktabs}
\newcommand{\PreserveBackslash}[1]{\let\temp=\\#1\let\\=\temp}
\usepackage[NewCommands]{ragged2e} % Better raggedleft formating
\newcolumntype{v}[1]{>{\PreserveBackslash{\raggedright}\hspace{0pt}}p{#1}}
\renewcommand{\arraystretch}{1.20}
\usepackage{multirow}

%\usepackage{minted}

\definecolor{bg}{rgb}{0.95,0.95,0.95}

\title[fn-fx]{GUI-Programmierung mit fn-fx}
\author{Chris Rutenkolk} 
\institute[HHU]{Institut f�r Informatik\\Lehrstuhl f�r Softwaretechnik und Programmiersprachen}
\date[06.06.2019]{6. Juni 2019}
\logo{\pgfimage[width=2cm]{../bilder/hhulogo}}

\begin{document}
%
\begin{frame}[plain]
  \begin{center}
    \includegraphics[width=4cm]{../bilder/hhulogo}
  \end{center}
  \vspace{-1cm}
  \titlepage
\end{frame}
%
\section*{Einleitung}
\setcounter{framenumber}{0}
%
\begin{frame}
    \frametitle{Es war einmal}
    \begin{itemize}[<+->]
        \item AWT
        \item Swing
        \begin{itemize}[<+->]
            \item SeeSaw
        \end{itemize}
        \item JavaFX (OpenJFX11) 
        \begin{itemize}[<+->]
            \item fn-fx (you are here)
        \end{itemize}
   \end{itemize}
\end{frame}
%
\begin{frame}
    \frametitle{GUI-Entwicklung}
    \begin{itemize}[<+->]
        \item SceneGraph
        \begin{itemize}[<+->]
            \item Nodes mit Children
            \item Praktisch �berall die Grundlage
            \item Aufbau durch stetige Mutation
            \item Nicht sehr praktisch / simpel 
        \end{itemize}
        \item Besser: Hierarchische DSL
        \begin{itemize}[<+->]
            \item Web / Javascript SPA: HTML
            \item JavaFX: FXML
        \end{itemize}
        \item Nachtr�gliche �nderungen?
        \begin{itemize}[<+->]
            \item JavaFX: SceneGraph Mutation
            \item Js/Web: DOM Mutation
            \item Angular,React,etc: nein, Virtual-DOM!
        \end{itemize}
    \end{itemize}
\end{frame}
%
\begin{frame}
    \frametitle{Idealfall}
    \begin{itemize}[<+->]
        \item MVC
        \item Model: Daten 
        \item View: Repr�sentierung
        \item Controller: Mutation
   \end{itemize}
\end{frame}
%
\begin{frame}
    \frametitle{Idealfall}
    \begin{itemize}
        \item MVC
        \item Model: Daten 
        \item View: Repr�sentierung
        \item Controller: \sout{Mutation} �nderung �ber die Zeit hinweg
   \end{itemize}
\end{frame}
\begin{frame}
    \frametitle{GUI-Entwicklung}
    \begin{itemize}[<+->]
        \item Was Angular f�r JS gemacht hat, macht fn-fx f�r JavaFX
        \item Clojure
        \begin{itemize}[<+->]
            \item Model (Daten): Clojure
            \item View (Hierarchische DSL): Clojure
            \item Controller: Clojure
            \item \sout{Mutation?} �nderung �ber die Zeit hinweg: atoms/agents
        \end{itemize}
        \item Styling
        \begin{itemize}[<+->]
            \item CSS
            \item generiert durch Clojure
        \end{itemize}
    \end{itemize}
\end{frame}
%
\begin{frame}
    \frametitle{GUI-Entwicklung}
    \begin{itemize}[<+->]
        \item Demo-Time
    \end{itemize}
\end{frame}
%
\begin{frame}
  \frametitle{Mutation}
    \begin{itemize}[<+->]
        \item �ndere den Text im Label in der VBox (Wie?)
        \item irgendwie m�ssen wir den Graph durchlaufen
        \item Gib mir das 2.e Kind des ersten Kindes der Scene
        \begin{itemize}[<+->]
            \item Layout �ndern = App kaputt machen
        \end{itemize}
        \item Irgendwoher hab ich den Parent (VBox)
        \item 2.es Kind
        \begin{itemize}[<+->]
            \item Besser. Aber Problem nur auf lokales Problem degradiert. Nicht gel�st.
        \end{itemize}
        \item fx:id
        \item Java-FX �bernimmt den Lookup. Cool!
        \item fx:ids \& @FXML inject nur via fxml
        \item Was wenn wir die Node komplett austauschen wollen?
    \end{itemize}
\end{frame}
%
\begin{frame}
    \frametitle{Mutation}
    \begin{itemize}[<+->]
        \item Problem:
        \begin{itemize}[<+->]
            \item Wir wollen hierarchisch
            \item Wir wollen keinen manuellen lookup
            \item Wir wollen eigentlich nur den Text �ndern
            \item Wo die �nderung in der View stattfindet ist uns beim �ndern der Daten egal
            \item (in dem Label. in der Vbox. In der Szene.)
        \end{itemize}
        \item kompletten Graph neu berechnen
        \begin{itemize}[<+->]
            \item Aber wann genau?
        \end{itemize}
        \item Virtueller Szenengraph, aus dem man den echten bauen kann
        \begin{itemize}[<+->]
            \item Leichtgewichtig
            \item Diff mit dem momentanen Szenengraph
            \item Austatusch von Nodes / Properties durch fn-fx
        \end{itemize}
    \end{itemize}
\end{frame}
%
\begin{frame}
    \frametitle{Recap}
    \begin{itemize}[<+->]
        \item Prozedural den Graph manipulieren
        \item vs.
        \item Einfach den virtual-DOM neu berechnen
    \end{itemize}
\end{frame}
%
\begin{frame}
    \frametitle{Bunte Dinge}
    \begin{itemize}[<+->]
        \item Entspannung
        \item Demo Time
    \end{itemize}
\end{frame}
%
\begin{frame}
  \frametitle{Funktionsweise}
    \begin{itemize}[<+->]
        \item Achtung: camel $\rightarrow$ kabob
        \item AnchorPane $\Rightarrow$ anchor-pane
        \item defui definiert User-Komponenten
        \item fn-fx.controls enth�lt JavaFX Controls Wrapper
        \begin{itemize}[<+->]
            \item Setze properties einer JavaFX Node durch
            \item nodeInstance.setMyProperty(myvalue) $\Rightarrow$ :my-property myvalue
            \begin{itemize}[<+->]
                \item \colorbox{black}{\lstinline[breaklines=true, basicstyle=\ttfamily\color{white}]|
                    myLabel.setStyle("-fx-background-color: red;")|}
                \item \colorbox{black}{\lstinline[breaklines=true, basicstyle=\ttfamily\color{white}]|
                    (controls/label :style "-fx-background-color: red;")|}
            \end{itemize}
            \item CrazyClassName.callMethod(this,value) $\Rightarrow$ :crazy-class-name/call-method myvalue
            \begin{itemize}[<+->]
                \item \colorbox{black}{\lstinline[breaklines=true, basicstyle=\ttfamily\color{white}]|
                    AnchorPane.setTopAnchor(this,0)|}
                \item \colorbox{black}{\lstinline[breaklines=true, basicstyle=\ttfamily\color{white}]|
                    (controls/label :anchor-pane/top-anchor 0.0)|}
            \end{itemize}
        \end{itemize}
    \end{itemize}
\end{frame}
%
\begin{frame}
  \frametitle{Funktionsweise}
    \begin{itemize}[<+->]
        \item Events sind Daten
        \item Bei Eventhandlern: Die map spezifizieren, die auf der ``Controller'' Seite ankommen soll
        \begin{itemize}[<+->]
                \item \colorbox{black}{\lstinline[breaklines=true, basicstyle=\ttfamily\color{white}]|
                    :on-click \{:foo :bar\}|}
                \item \colorbox{black}{\lstinline[breaklines=true, basicstyle=\ttfamily\color{white}]|
                    :on-action \{:event-type :the-red-button-was-pressed\}|}
        \end{itemize}
        \item Informationen mit Event mitschicken:
        \item :fn-fx/include
        \begin{itemize}[<+->]
                \item \colorbox{black}{\lstinline[breaklines=true, basicstyle=\ttfamily\color{white}]|
                     :on-action \{|}
                     \par
                      \colorbox{black}{\lstinline[breaklines=true, basicstyle=\ttfamily\color{white}]|
                      :event :foo|}                   
                      \par
                \begin{itemize}[<+->]
                        \item 
                      \colorbox{black}{\lstinline[breaklines=true, basicstyle=\ttfamily\color{white}]|
                      :fn-fx/include \{:my-dom-node-id \#\{:text :foo-property\}|}
                        \item 
                      \colorbox{black}{\lstinline[breaklines=true, basicstyle=\ttfamily\color{white}]|
                      :fn-fx/include \{:fn-fx/event \#\{:target :bar\}\}|}
                \end{itemize}
        \end{itemize}
    \end{itemize}
\end{frame}
%
\begin{frame}
  \frametitle{Kritik}
    \begin{itemize}[<+->]
        \item Dokumentation
        \begin{itemize}[<+->]
            \item Magisches Verhalten u. Features
            \begin{itemize}[<+->]
                \item fn-fx/includes map: keys werden per find-nearest-by-id als lookup f�r dom-nodes genutzt
                \item vgl. form-demo. Wie kommen in das Event die Eingaben aus den anderen Feldern?
                \item Cooles Feature. H�tte ich gerne von Anfang an gewusst
            \end{itemize}
            \item Magische genamespacete Keywords (welche? wann?)
            \begin{itemize}[<+->]
                \item fn-fx/includes
                \item fn-fx/event
                \item fn-fx/listen
                \item class-name/static-method
            \end{itemize}
        \end{itemize}
        \item Nicht ganz Feature-Complete
        \begin{itemize}[<+->]
                \item Property-Bind
                \item fn-fx/bind (?)
        \end{itemize}
    \end{itemize}
\end{frame}
%
\begin{frame}
  \frametitle{Aufgabe f�r Heute}
    \begin{itemize}[<+->]
        \item Kleinere Tasks zum warm werden
        \begin{itemize}[<+->]
            \item VBox mit Button und Label
            \begin{itemize}[<+->]
                \item VBox mit Button und TextField
                \item Buttonpress $\Rightarrow$ dupliziere TextField Text
            \end{itemize}
            \item Layouting
            \begin{itemize}[<+->]
                \item GridPane Form
            \end{itemize}
            \item CSS
            \begin{itemize}[<+->]
                \item garden
                \item css.clj Anpassen
            \end{itemize}
    \end{itemize}
        \item Etwas mehr
        \begin{itemize}[<+->]
            \item Wetter-App mit TextEingabe etc.
            \item z.B. via http://wttr.in
            \item (slurp "http://wttr.in/d�sseldorf?format=\%l:\%C,\%t")
        \end{itemize}
    \end{itemize}
\end{frame}
%
\begin{frame}[plain]
  \frametitle{Fragen}
	\begin{center}
 		Vielen Dank f�r Ihre Aufmerksamkeit!
	\end{center}
\end{frame}
\end{document}
